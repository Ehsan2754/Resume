\section{Skills}
    


    \subsection{\textbf{Stack/Tools}}
    
        \cvitem{Embedded middlewares}{ ZeroMQ, GigEVision, GNSS NMEA-0183, openCyphal, dronecanV0.9, UAVCANv0, FreeRTOS, EMR}
        \cvitem{LL-Protocols}{UART/USART, I2C/SMBUS, SPI, BLE, CAN, MIPI}
        \cvitem{ARM cortex-Mx}{GNU-ARM-Toolchain,CMSIS, baremetal, openOCD}
        \cvitem{Embedded AI}{TinyML. Edge impulse. STM32AI cube, ONXX}
        \cvitem{Machine Learning}{ pyTorch, Scikit learn, gymnasium, OpenCV2, tensorflow2, tf.lite, nltk}
        \cvitem{Database}{posgresSQL, SQlite, MongoDB, Redis, MS-Access}
        % \cvitem{DevOps/BigData}{Hadoop, HDFS, Yarn, Docker}
        \cvitem{CI/CD}{Github Actions, Yandex Cloud, TravisCI, Heroku, negrok}
        \cvitem{PCB-layout/Scheme}{Altium Designer, Proteus, KiCad}
        \cvitem{CAD}{SolidWorks}
        \cvitem{Version Control}{Git, Github}
        \cvitem{Task management}{Slack, Notion, Trello, Bitrix24, Jira}
        \vspace*{5pt}

        \subsection{\textbf{Programming Languages/Libraries/Frameworks}}
        \cvitem{C}{stdc, linux-kernel, network, POSIX,  RPC, sockets, TinyML, openmp}
        \cvitem{C++}{OpenCL, STL, Gtest(Google Test), BoostASIO,  AsyncTCP, AsyncWebWerver, tflite}
        \cvitem{Python}{ pandas, numpy, XML/Json RPC, Flask, SQLalchemy, Redis, flask-restful, scrappy/spiders, beautifulSoup, paramiko, socket, twine, ipython, virtualenv, Sphinx}
        
        \cvitem{Java}{apache.spark}
        \cvitem{C\#}{core.Net 4, ADO.net}
        \cvitem{Other(Hobby)}{  Tact \textit{ for TON Blockchain },  Rust (Rocket),  Next.js, Verilog, Julia, Haskell }
    % {
    % \begin{itemize}
    %     \item[*] I developed my skills in Rust programming as a hobby through working on an open-source project called \href{https://github.com/objectionary/reo}{\textcolor{blue}{EO-RUST transpiler(Repo)}} , which was led by \href{https://github.com/yegor256}{\textcolor{blue}{yegor256}} and sponsored by Huawei. 
    %     \item[*] I had the opportunity to work with Julia on some basic reinforcement learning (RL) experimentation using various framework  \href{https://github.com/Ehsan2754/FunWithScience/tree/main/notebooks/.jl}{\textcolor{blue}{playgrounds}}.
    %     \item[*] I acquired knowledge and skills in MIPS-32 Assembly Language and Verilog through participating in workshops on hardware design as part of a Joint-Inspection-Group.
    % \end{itemize}
    % }
    \vspace*{5pt}
        
        
    \subsection{\textbf{Languages}}
        \cvlanguage{Persian}{Native Language}{}
        \cvlanguage{English}{Primary Communication Language }{}%{\textbf{C1}}
        \cvlanguage{Russian}{Foreign Language}{}%{\textbf{B1}}
        \cvlanguage{Arabic}{Foreign Language}{}%{\textbf{B1}}
        \cvlanguage{German}{Foreign Language}{}
        \vspace*{5pt}
% \section{Interests}
%     \cvitem{Data Collection}{Crawling, Labeling, Noise Channel Modeling}
%     \cvitem {Data Analysis}{ EDA, Data Analysis, Dimension Reduction} 
%     \cvitem{Data Prediction}{ Feature detection, Regression Models, Classification Techniques, CNN, LSTM, RNN}
%     \cvitem {Development}{ Data-Collection beacons, High-Performance Programming, Low-level Module Development, Low Latency Programming}
%     \cvitem {Management}{Pipeline Designing, Code Review, Backlog management  }
